\documentclass[A4,10pt]{article}
\usepackage[dvips]{graphicx}
\usepackage[spanish]{babel}
%\selectlanguage{spanish}
\usepackage[utf8]{inputenc}
\usepackage{amsthm,amsmath,amssymb,graphics,mathrsfs}
\usepackage[active]{srcltx}
\usepackage{hyperref}
\usepackage{cite}
\usepackage{multicol,pstricks,pstricks-add,setspace}
%\usepackage[margin=10pt,font=footnotesize,labelfont=bf,labelsep=endash]{caption}
\usepackage[left=4cm,top=3cm,right=2.4cm,bottom=3.2cm]{geometry}
% \newcolumntype{E}{>{$}c<{$}}
\usepackage{pst-all}
\usepackage{nonfloat}
\usepackage[bf,footnotesize]{caption2}
\usepackage[dvips]{graphicx}
\pagestyle{myheadings}
\parindent=0pt
%opening
\usepackage{fancyhdr}
\pagestyle{fancy}
\fancyhead{}
\fancyhead[R]{Proyecto Modelado Matemático I}
\fancyfoot{}
\fancyfoot[L]{López, J. \textit{E-mail: \href{jorge2218061@correo.uis.edu.co}{jorge2218061@correo.uis.edu.co}}}
\fancyfoot[R]{\thepage}
\renewcommand{\headrulewidth}{0pt}
\title{{\huge \textbf{Buscando soluciones numéricas vía Python de EDPs: Una introducción de las EDPs en una y dos dimensiones que conducen a las ecuaciones de Navier-Stokes.}}}
\author{Jorge Leonardo López Agredo\\ 
	\small Universidad Industrial de Santander Escuela de Física\\
	 \small jorge2218061@correo.uis.edu.co\\ 
	\small Maestría en Matemática Aplicada\\
	 Docente: Ph.D Juan Carlos Basto Pineda\\ 
	\date{}
 }

\begin{document}

	\maketitle
	\vspace*{-1.5cm}
\begin{center}\rule{0.9\textwidth}{0.1mm} \end{center}

\section{Objetivo general}

Construir una actividad pedagógica para el curso de Modelado Matemático $I$, en el cual se introduzca las ideas básicas  de  los esquemas de diferencias finitas hacia adelante y hacia atrás, relacionadas con las soluciones numéricas de las ecuaciones diferenciales parciales vía Python.
\subsection{Objetivos específicos:}
\begin{enumerate}
	\item Crear un Jupyter Notebook donde se aborden la construcción de los esquemas de diferencias finitas que permita solucionar numéricamente una ecuación diferencial parcial en una dimensión.

\item Desarrollar una actividad pedagógica que permita estudiar la ecuaciones en una dimensión necesarias para introducir las ecuaciones de Navier-Stokes.

\item Introducir algunas ideas básicas para la construcción de un método numérico que permita afrontar el problema en dos dimensiones de las ecuaciones relacionadas con las ecuaciones de Navier-Stokes.

\item  Diseñar un ejercicio final, donde los estudiantes puedan implementar las ideas desarrolladas en la actividad pedagógica, usando las herramientas aprendidas en el curso sobre Python y el esquema de diferencias finitas.

	
\end{enumerate}


%	\vspace*{-2.0cm}
\begin{center}\rule{0.9\textwidth}{0.1mm}
 \end{center}
\begin{multicols}{2}


\section{Introducción}

Las ecuaciones de Navier-Stokes (ENS) son un conjunto de ecuaciones diferenciales parciales (EDP) no lineales que describen el movimiento de un fluido viscoso, las cuales expresan matemáticamente la conservación de momento  y de la masa para los fluidos newtonianos (fluido cuya viscosidad puede considerarse constante)\cite{Achenson}.  Su nombre se debe en honor del físico e ingeniero francés Claude-Louis Navier (1785-1836) y al físico-matemático irlandés George Gabriel Stokes (1819-1903), los cuales desarrollaron la formulación integral y diferencial aplicando los principios de conservación de la mecánica y la termodinámica a un volumen de fluido y ciertas consideraciones como lo fue los esfuerzos tangenciales y su relación lineal con el gradiente de velocidad del movimiento del fluido, conocida como Ley de viscosidad de Newton\cite{Achenson}.\\


De las ENS no se dispone de una solución general de manera analítica, salvo para casos particulares, en fluidos específicos y bajo condiciones muy concretas\cite{Girault,Achenson}. Es por esto, que es preciso recurrir al análisis numérico, para determinar una solución aproximada que permita modelar problemas más generales. La dinámica de fluidos computacional (CFD) es la rama de la mecánica de fluidos que se ocupa del estudio mediante métodos numéricos de soluciones\cite{Barbagroup,Girault}.


\end{multicols}
\begin{thebibliography}{99}
	\addcontentsline{toc}{chapter}{Bibliografía}
	\bibitem{Barba} Barba, Lorena A., and Forsyth, Gilbert F. (2018). \textit{CFD Python: the 12 steps to Navier-Stokes equations}. Journal of Open Source Education, 21. [\href{https://doi.org/10.21105/jose.00021}{https://doi.org/10.21105/jose.00021}]
%\bibitem{sch1} SCHRÖDINGER , E. \textit{Annalen der Physik ILXXX}. Folge 361, 1926.
%\bibitem{opticalphysics} LIPSON, S. G.; LIPSON, H. \textit{Optical Physics.} 2 ed. New York: Cambridge University Press, 1981.
%\bibitem{mediblefuncion} S. LUNDEEN, J. S.; SUTHERLAND, B.; PATELL, A.;  STEWART C. \&  BAMBERT, C. \emph{Direct measurement of the quantum wavefunction.} Nature 474, (2011). p. 188-191.
\bibitem{Barbagroup} Barbagroup, (2019). \textit{CFDPython}. [\href{https://github.com/barbagroup/CFDPython}{https://github.com/barbagroup/CFDPython}]
\bibitem{Achenson} Achenson, D. J. (1990). \textit{Elementary Fluid Dynamics, Oxford Applied Mathematics and Computing Science Series}. Oxford University Press, ISBN 0-19-859679-0.
\bibitem{Girault} V. Girault and P.A. Raviart. \textit{Finite Element Methods for Navier–Stokes Equations: Theory and Algorithms}. Springer Series in Computational Mathematics. Springer-Verlag, 1986.
\end{thebibliography}
\end{document}
